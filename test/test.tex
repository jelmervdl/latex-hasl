\documentclass[a4paper]{article}
\usepackage[utf8]{inputenc}
\usepackage[pdf,tmpdir]{haslgraph}

\pdfsuppresswarningpagegroup=1

\begin{document}
\title{HASL argument diagrams}
\author{Jelmer van der Linde}
\date{\today}

\maketitle

\newpage

\begin{figure}
	\begin{haslpicture}[width=0.5\textwidth]{argJohn}
		a: John is a doctor
		b: John graduated
		c: John finished his thesis
		d: b supports a
		e: c supports b
	\end{haslpicture}
	\caption{Argument diagram for sentence with shared claim/minor.}
	\label{fig:argJohn}
\end{figure}

\begin{figure}
	\begin{haslpicture}[width=0.5\textwidth]{argtcanfb}
		a: Tweety can fly
		b: She is a bird
		c: assume Birds can fly
		d: b supports a
		e: c supports d
	\end{haslpicture}
	\caption{Argument diagram for sentence with implicit major.}
	\label{fig:argTcanFB}
\end{figure}

\begin{figure}
	\begin{haslpicture}[width=0.5\textwidth]{argBirds}
		a: Tweety can fly
		b: Tweety is a bird
		f: Birds can fly
		d: b supports a
		g: f supports d
		c: Birds have wings
		e: c supports g
	\end{haslpicture}
	\caption{Argument diagram for sentence with supported major.}
	\label{fig:argBirds}
\end{figure}

\begin{figure}
	\centering
	\begin{haslpicture}[width=\textwidth]{argtoulmin}
		a: Claim
		b: Datum
		w: Warrant
		c: b supports a
		h: w supports c
		d: Backing
		e: d supports w
		f: Rebuttal
		g: f attacks h
	\end{haslpicture}
	\caption{Toulmin's model}
	\label{fig:argtoulmin}
\end{figure}

\begin{figure}
	\centering
	\begin{haslpicture}[width=\textwidth]{rules}
		a: Tweety is a bird
		b: Tweety can fly
		c: Birds can fly
		d: It is a bird
		e: It is a penguin
		1: a supports b
		2: c supports 1
		3: d warrants c
		4: e undercuts 3
	\end{haslpicture}
	\caption{Explicit rules}
	\label{fig:argrules}
\end{figure}

\begin{figure}
	\centering
	\begin{haslpicture}{rbc2r}
		w: Warrant
		c1: C1
		c2: C2
		a: c1 c2 warrants w
	\end{haslpicture}
	\caption{Compound statements are no longer drawn}
\end{figure}

\begin{figure}
	\centering
	\begin{haslpicture}{poundsign}
		a: Tweety (#1) is a bird
		b: He (#1) can fly
		c: Henry (#2) said he (#1) is
		1: b supports a
		2: c supports a
	\end{haslpicture}
	\caption{Pound signs are not duplicated}
\end{figure}

\begin{figure}
	\centering
	\begin{haslpicture}{errortest1}
		a: Tweety (#1) is a bird
		b: He (#1) can fly
		c: Henry (#2) said he (#1) is
		1: d supports a
		2: c supports a
	\end{haslpicture}
	\caption{Intentional error (d undefined variable)}
\end{figure}

\begin{figure}
	\centering
	\begin{haslpicture}{errortest2}
		a: Tweety (#1) is a bird
		b: He (#1) can fly
		c: Henry (#2) said he (#1) is
		1: d supports a
		2: c supports a
		d: I wont be connected
	\end{haslpicture}
	\caption{Intentional error (d undefined variable)}
\end{figure}

\begin{figure}
	\centering
	\begin{haslpicture}{wraptest}
		style claim.maxWidth 100
		a: This is a very long text about Tweety (#1) being a bird
		b: He (#1) can fly
		c: Henry (#2) said he (#1) is during the festival we both attended in the summer of '78 while we were still popsticles
		1: b supports a
		2: c supports a
	\end{haslpicture}
	\caption{Wrap test}
\end{figure}

\begin{figure}[h]
	\centering
	\begin{haslpicture}[scale=0.25]{hasl1evalconjunctiontort}
		a: John must repair
		b: Jack suffered damage
		c: The damage was caused by an act of John
		d: The act was unlawful
		e: The act is imputed to John
		f: This article of law describes it as such
		g: The opintion describes it as such
		h: b c d e supports a
		i: f supports e
		j: g supports e
	\end{haslpicture}
	\caption{Too wide argument.}
	\label{fig:hasl1evalconjunctiontort}
\end{figure}

\begin{figure}[h]
	\centering
	\begin{haslpicture}[scale=0.5]{multiplewarrantstest}
		a: Tweety can fly
		b: Tweety is a bird
		c: something can fly if something is a bird
		d: Tweety is a squirrel
		e: b supports a
		f: c supports e
		g: d attacks e
		i: Tweety is a dog
		h: i attacks e
	\end{haslpicture}
	\caption{Too wide argument.}
	\label{fig:hasl1evalconjunctiontort}
\end{figure}



\end{document}