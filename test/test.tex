\documentclass[a4paper]{article}
\usepackage[utf8]{inputenc}
\usepackage[pdf,tmpdir]{haslgraph}

\begin{document}

\title{HASL argument diagrams}
\author{Jelmer van der Linde}
\date{\today}

\maketitle

\newpage

\begin{figure}
	\haslgraph[width=0.5\textwidth]{argJohn}{
		a: John is a doctor
		b: John graduated
		c: John finished his thesis
		d: b supports a
		e: c supports b
	}
	\caption{Argument diagram for sentence with shared claim/minor.}
	\label{fig:argJohn}
\end{figure}

\begin{figure}
	\haslgraph[width=0.5\textwidth]{argtcanfb}{
		a: Tweety can fly
		b: She is a bird
		c: b supports a because assume Birds can fly
	}
	\caption{Argument diagram for sentence with implicit major.}
	\label{fig:argTcanFB}
\end{figure}

\begin{figure}
	\haslgraph[width=0.5\textwidth]{argBirds}{
		a: Tweety can fly
		b: Tweety is a bird
		d: b supports a because birds can fly
		c: Birds have wings
		e: c supports d
	}
	\caption{Argument diagram for sentence with supported major.}
	\label{fig:argBirds}
\end{figure}

\begin{figure}
	\centering
	\haslgraph[width=\textwidth]{argsocdeadoralive}{
		a: Socrates is dead
		b: His heart is beating
		c: a attacks b because assume When someone dies their heart stops beating
		d: b attacks a because assume When someone's heart is beating they are alive
	}
	\caption{Argument diagram for sentence with reflexive counterclaim.}
	\label{fig:argsocdeadoralive}
\end{figure}

\begin{figure}
	\centering
	\haslgraph[width=\textwidth]{argtoulmin}{
		a: Claim
		b: Datum
		w: Warrant
		c: b supports a
		h: w supports c
		d: Backing
		e: d supports w
		f: Rebuttal
		g: f attacks h
	}
	\caption{Toulmin's model.}
	\label{fig:argtoulmin}
\end{figure}

\begin{figure}
	\centering
	\haslgraph[width=\textwidth]{rules}{
		a: Tweety is a bird
		b: Tweety can fly
		c: Birds can fly
		d: It is a bird
		e: It is a penguin
		1: a supports b
		2: c supports 1
		3: d warrants c
		4: e undercuts 3
	}
	\caption{Explicit rules}
	\label{fig:argrules}
\end{figure}

\begin{figure}
	\centering
	\haslgraph{rbc2r} {
		w: Warrant
		c1: C1
		c2: C2
		a: c1 c2 warrants w
	}
\end{figure}

\end{document}